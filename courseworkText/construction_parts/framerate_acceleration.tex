\section {Способ увеличения частоты кадров} {
    Пусть в задаче визуализации сцены уже построены буферы относительно источников
    и необходимо трассировать лучи для получения цвета каждого пикселя.
    Тогда можно сначала создать изображение размерность когорого меньше размерности
    исходного в $k$ раз, вычисляя цвет пикселей с шагом $k$.
    Затем полученное изображение необходимо отмасштабировать под размер ожидаемого.
    Таким образом, экономится время работы.
    На следующем шаге происходит уменьшение k в 2 раза и расчёт остальных пикселей
    для генерации кадра следующей размерности.
    В результате пользователь может увидеть сначала нечёткое изображение сцены,
    а затем полностью сгенерированное изображение.
}