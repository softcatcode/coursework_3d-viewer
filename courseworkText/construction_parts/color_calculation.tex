\section{Разделение луча} {
    Рассматривается следующая ситуация: луч падает на точку объекта,
    разделяясь на преломлённый и отражённый.
    Сначала необходимо определить способ поиска
    направления отражённого и преломлённого лучей.
    Пусть:
    \begin{itemize}
        \item $\vec{n}$ -- направление отражённого луча;
        \item $\vec{r}$ -- направление преломлённого луча;
        \item $\vec{R}$ -- направление отражённого луча;
        \item $\vec{T}$ -- направление луча после преломления.
    \end{itemize}
    Тогда выполнены равенства \ref{f:R} и \ref{f:T}:
    \begin{equation}
        \label{f:R}
        \vec{R} = [\vec{n} \times [\vec{r} \times \vec{n}]]
    \end{equation}
    \begin{equation}
        \label{f:T}
        \vec{T} = R \cdot \frac{k_1}{k_2}
        \cdot (\vec{R}; \vec{r}) - \vec{n} \cdot
        \sqrt{ 1 - (\frac{k_1}{k_2} \cdot (\vec{R}; \vec{r}))^2 }
    \end{equation}
    
    Яркость луча определяется интенсивностью красной, зелёной и синей соствыляющих, каждая из которых может быть
    вещественным числом, большим 255.
    Для реализации трассировки необходимо реализовать алгоритм распределения
    интенсивности света между данными сегментами луча.
    Этот алгоритм должен удовлетворять следующим условиям:
    \begin{itemize}
        \item
            чем меньше коэфициент пропускания объекта,
            тем больше света пройдёт сквозь него;
        \item
            от объекта не может отразиться луч с составляющей яркости
            большей соответствующей составляющей цвета объекта.
    \end{itemize}
    Они следуют из физических соображений.
    Пусть $k_t, k_d, k_s$ -- коэфициенты пропускания,
    диффузного и зеркального отражения соответственно.
    Далее представлены шаги алгоритма для одной составляющей яркости $x$:
    \begin{itemize}
        \item вычисляется пропущенная яркость: $x_{tr} := x \cdot k_t$;
        \item вычисляется отражённая составляющая: $x_r := x - x_tr$.
    \end{itemize}
    В результате составляющие яркости падающего луча распределяются
    между преломлённым и отражённым лучами.
    Итоговый цвет, который увидет наблюдатель в данной точке, вычисляется
    как среднее арифметическое из цветов лучей, на которые распался данный.
}