\section{Описание алгоритма с картой теней} {
    Целью данного раздела является
    подробное описание выбранного алгоритма генерации теней.
    Это необходимо для разработки программы, визуализирующей сцену.
    Важным замечанием является то, что источник света исеет свою систему координат,
    причём направление освещения совпадает с направлением оси $z$;
    аналогично наблюдатель (камера), имея собственную систему координат,
    смотрит в направлении своей оси $z$.
    Далее описаны способы выполнения основных шагов алгоритма
    затенения с помощью отражательной карты теней.
    \subsection{Построение буфера источника} {
        Пусть прямая $a$ параллельна оси $z$ источника и содержит
        точки с координатами $(j, i, z), z \in \mathbb{R}$.
        Под буфером источника подразумевается матрица размерности изображения, которая
        в элементе на позиции $(i, j)$ хранит массив структур, содержащих данные о
        пересечениях $a$ со сценой.
        Их можно рассматривать как столкновение луча, прошедшего через сцену, с точкой.
        В данную структуру входит [2]:
        \begin{itemize}
            \item цвет пикселя в формате RGB;
            \item глубина точки сцены;
            \item внешняя нормаль к поверхности объекта в точке,
            соответствующей данному элементу;
            \item суммарный поток света, достинший точки от всех источников;
            \item координаты точки, соответствующей данному пикселю относительно
            системы координат сцены;
            \item свойства объекта;
            \item точность сравнения глубин точек
            при определении факта затенения.
        \end{itemize}
        Далее описан способ построение такого буфера относительно одного источника.
        Сначала координаты объектов сцены преобразуются к системе координат источника.
        Затем перебираюися точки проекции каждой грани на плоскость $(xy)$,
        имеющие целочисленные координаты.
        Для каждой такой точки собирается описанная выше структура.
        Когда все точки перебраны, происходит сортировка по глубине массива
        структур каждого элемента буфера.
        На посдеднем шаге выполняется проход по каждому такому массиву для того, чтобы
        расчитать яркость для каждой его структуры.
    }
    \subsection{Определение яркости пикселя} {
        Под буфером подразумевается буфер глубины, элементы которого содержат не только
        саму глубину, но и дополнительную информацию,
        необходимую для создания карты теней [2].
        Пусть уже построены буферы глубины относительно всех источников.
        Яркость пикселя делится на две составляющие: от прямого освещения и от
        отражений света от поверхности объектов.
        Далее рассмотрены спомобы вычисления каждой из этих составляющих.
        \subsection{Вычисление яркости от прямого освещения} {
            Пусть $\vec{P}(x, y, z)$ -- радиус-вектор точки сцены $P$,
            для которой нужно расчитать яркость.
            Яркость от прямого освешения равна сумме интенсивностей света, достигших
            точки от каждого источника [2].
            Поэтому рассмотрено нахождение вклада в яркость только одним источником,
            буфер $buf$ которого пусть уже построен.
            $T$ -- матрица перехода к системе координат источника.
            $\vec{P}'(j, i, depth) = T \times \vec{P}$.
            Рассматривается элемент буфера $e$ на позиции $(i, j)$
            с целью вычислить яркость $P$.
            Затем перебираются структуры элемента $e$, пока не будет встречена та,
            значение глубины в которой примерно равно $depth$.
        }
        \subsection{Вычисление яркости от отражений} {
            Чтобы найти яркость от диффузного отражения для точки сцены $P$,
            нужно перебрать все элементы буфера каждого источника.
            В каждом элементе выбирается одна или несколько структур,
            каждая из которых вносит вклад в результирующую яркость.
            Далее приведена формула вычисления этого вклада [2]. \\
            Пусть:
            \begin{itemize}
                \item $x$ -- точка сцены, для которой расчитывается дополнительная яркость;
                \item $x_p$ -- точка сцены, которая делает вклад в яркость $x$;
                \item $\vec{n_p}, \vec{n_x}$ -- вектора нормали в $p, x$ соответственно;
                \item $\vec{\omega}$ -- вектор, соединяющий $p$ с $x$;
                \item $I_p(\vec{\omega})$ -- интенсивность, испускаемая точкой $p$ в
                направлении $\vec{\omega}$.
                \item $\Phi$ -- поток света, достигший от источнока до $p$ (хранится в RSM).
            \end{itemize}
            Тогда выполнено равенство \ref{f:I_p} [2].
            \ref{f:I_p}.
            \begin{equation}
                \label{f:I_p}
                I_p(\vec{\omega}) = k\Phi max\{0, (\vec{n_p}, \vec{\omega})\}
            \end{equation}
            Пусть:
            \begin{itemize}
                \item $E_p(x, p)$ -- вклад точки $p$ в яркость точки $x$;
                \item
                    $k$ -- коэфициент диффузного отражения,
                    который представлен в задании;
                \item разностью точек $A - B$ является вектор $\vec{AB}$.
            \end{itemize}
            Тогда справедлива формула \ref{E_p} [2].
            \begin{equation}
                \label{f:E_p}
                E_p(x, p) = k\Phi\frac{ max\{0, (n_p, x - x_p)\}
                max\{0, (n_x, x_p - x)\} }
                {|x - x_p|^4}
            \end{equation}
            Значение дополнительной яркости точки засчёт отражений
            обозначено как $E(x, p)$ и вычисляется по формуле \ref{f:E}~[2].
            \begin{equation}
                \label{f:E}
                E(x, p) = \sum_p E_p(x, p)
            \end{equation}
        }
        \subsection{Выбор элементов буфера, вносящих вклад в освещение от отражений} {
            Пусть необходимо расчитать яркость для точки $P$, которой соответствует
            элемент буфера источника на позиции $(i_0, j_0)$.
            Тогда те элементы этого буфера, которые находятся около данного,
            соответвтвуют точкам сцены, которые, скорее всего, обизки к $P$.
            Этот факт указан в статье, посвещённой данному алгоритму.
            Таким образом, следует в основном выбирать те элементы буфера,
            которые близки к позиции $(i_0, j_0)$.
            Следует производить выбор очередных индексов $(i, j)$
            по следующему алгоритму [2]:
            \begin{itemize}
                \item
                    генерируется 2 случайных вещественных числа
                    $\alpha, \beta$ из нормального распределения от 0 до 1;
                \item $i := i_1 + r_{max} \cdot \mathbf{\alpha} \sin(2\pi\beta)$;
                \item $j' := j_1 + r_{max} \cdot \mathbf{\alpha} \cos(2\pi\beta)$.
            \end{itemize}
            Таким образом, имеем элемент буфера источника на позиции $(i, j)$,
            который будет участвовать в расчёте яркости от отражений для точки $P$.
        }
    }
}