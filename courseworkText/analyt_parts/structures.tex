\section{Используемые структуры} {
    \subsection{Массив} {
        Массив -- структура данных, содержащая однотипные элементы,
        располложенные в памяти последовательно.
        Обращение к элементу осуществляется по его индексу.
        Индекс элемента -- целое неотрицательное число.
        Одна и та же структура массива будет использована для хранения
        элементов любых типов.
    }
    \subsection{Maтрица} {
        Матрица -- массив $a$, элементами которого являются другие массивы.
        Обращение к элементу матрицы осуществляется по 2 индексам:
        номеру строки и столбца элемента.
        Номер строки -- индекс массива $a$.
        Номер столбца -- индекс элемента в определённом массиве из $a$.
        Одна и та же структура матрицы будет использована для хранения
        элементов любых титпов.
    }
    \subsection{Вектор и точка} {
        Представляет собой структуру $Vector$ из  3 вещественных чисел --
        координат математисеского вектора в пространстве.
        Точка в пространстве, $Point$, также представляется своим радиус-вектором.
        Таким образом, с точкой можно будет работать так же, как и с вектором.
        Таким образом, описание полей и вектора, и точки выглядит так:
        \begin{enumerate}
            \item вещественное число $x$;
            \item вещественное число $y$;
            \item вещественное число $z$.
        \end{enumerate}
    }
    \subsection{Цвет} {
        Представляет собой 3 целых знаковых числа $r, g, b$, обозначающих интенсивность
        красного, синего и зелёного цветов соответственно.
        Поля данной структкры $Color$:
        \begin{enumerate}
            \item целое знаковое число $r$;
            \item целое знаковое число $g$;
            \item целое знаковое число $b$.
        \end{enumerate}
    }
    \subsection{элементы структуры RSM} {
        Данная структура представляет собой матрицу, содержащую элементы типа <<RSMElem>>.
        Каждый элемент содержит:
        \begin{enumerate}
            \item $color$ -- цвет пикселя;
            \item вещественное число $depth$ -- глубина точки;
            \item вектор нормали $n$ к точке сцены;
            \item вешественное число $brightness$ -- яркость точки сцены;
            \item $point$ -- точка сцены.
        \end{enumerate}
    }
    \subsection{Изображение} {
        Изображение $Image$ представляет собой матрицу, элементы которой
        являются объектами типа $Color$.
        Дополнительно хранятся размеры изображения.
        Поля структуры $Image$:
        \begin{enumerate}
            \item $batch$ -- матрица пикселей;
        \end{enumerate}
    }
    \subsection{Камера} {
        Данная стркутура представляет любого наблюдателя сцены.
        Список полей структуры $Camera$:
        \begin{enumerate}
            \item $z$ -- вектор напраления камеры;
            \item $x$ -- координатный вектор;
            \item $y$ -- координатный вектор;
            \item $location$ -- точка положения камеры.
        \end{enumerate}
    }
    \subsection{Источник света} {
        Данная структура представляет собой источник света, освещающий сцену.
        Список полей структуры $Source$:
        \begin{enumerate}
            \item $z$ -- вектор напраления освещения;
            \item $x$ -- координатный вектор;
            \item $y$ -- координатный вектор;
            \item $location$ -- точка положения источника
        \end{enumerate}
    }
    \subsection{Треугольная грань} {
        Список полей структуры $Triangle$:
        \begin{enumerate}
            \item беззнаковое целое число $a$ -- индекс точки, обозначающей первую вершину;
            \item беззнаковое целое число $b$ -- индекс точки, обозначающей вторую вершину;
            \item беззнаковое целое число $c$ -- индекс точки, обозначающей третью вершину;
            \item вектор нормали $n$ к данной грани.
        \end{enumerate}
    }
    \subsection{Сфера} {
        При трассировке лучей вокруг объектов описываются сферы.
        Поэтому требуется отдельная структура $Sphere$ со слкдующими полями:
        \begin{enumerate}
            \item $center$ -- точка, обозначающая центр;
            \item $radius$ -- радиус сферы.
        \end{enumerate}
    }
    \subsection{Объект} {
        Данная структура представляет собой любой объект сцены, который можно отрисовать.
        Список полей структуры $Object$:
        \begin{enumerate}
            \item $points$ -- массив точек;
            \item $triangles$ -- массив граней;
            \item $type$ -- тип источника.
        \end{enumerate}
    }
    \subsection{Сцена} {
        Объект $Stage$ представляет собой контейнер для хранения всех данных,
        необходимых и достаточных для рендера изображения.
        \begin{enumerate}
            \item $sources$ -- массив источников света;
            \item $objects$ -- массив объектов;
            \item $camera$ -- камера.
        \end{enumerate}
    }
    \section{Свойства объекта} {
        Поля структуры данных <<ObjectProperties>:
        \begin{enumerate}
            \item $color$ -- цвет;
            \item $transmission$ -- коэфициент пропускания;
            \item $reflection$ -- коэфициент отражения;
            \item $diffuseRefl$ -- коэфициент диффузного отражения;
            \item $optDensity$ -- оптическая плотность.
        \end{enumerate}
    }
    \section {Точка столкновения луча и объекта} {
        Поля структуры данных <<Collision>:
        \begin{enumerate}
            \item $x$ -- x координата;
            \item $y$ -- y координата;
            \item $z$ -- z координата;
            \item $objProp$ -- свойства объекта;
            \item $n$ -- нормаль;
            \item $brightness$ -- яркость точки;
            \item $delta$ -- доппустимая погрешность сравнения глубин.
        \end{enumerate}
    }
    \section {Трассировщик} {
        Поля структуры данных <<RayTracer>:
         \begin{enumerate}
            \item $data$ -- матрица лучей;
            \item $spheres$ -- сферы, описанные вокруг объектов;
            \item $transformers$ -- структуры для перехода к координатам источников;
            \item $sourceMaps$ -- буферы источников;
            \item $objects$ -- объекты;
            \item $sources$ -- источники света.
         \end{enumerate}
    }
    \section{Луч} {
        Луч представляет собой массив секментов луча <<BeamSegment>>.
        Поля структуры данных <<BeamSegment>>:
        \begin{enumerate}
            \item $direction$ -- направление луча;
             \item $origin$ -- исток луча;
            \item $color$ -- цвет в формате rgb;
            \item $power$ -- текущая интенсивность луча;
            \item $collisionCount$ -- количчество столкновений с объектами.
        \end{enumerate}
    }
    \section {Структура перехода} {
        Поля структуры данных <<Transformer>>:
        \begin{enumerate}
            \item $matrix$ -- матрица перехода к другой системе координат.
            \item $bias$ -- вектор, соединяющий начала новой и старой систем координат;
        \end{enumerate}
    }
}