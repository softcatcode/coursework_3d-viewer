\section{Используемые структуры данных} {
    Массив -- структура данных, содержащая однотипные элементы,
    расположенные в памяти последовательно.
    Обращение к элементу осуществляется по его индексу.
    Индекс элемента -- целое неотрицательное число.
    Одна и та же структура массива будет использована для хранения
    элементов любых типов.
    
    Матрица -- массив $a$, элементами которого являются другие массивы.
    Обращение к элементу матрицы осуществляется по 2 индексам:
    номеру строки и столбца элемента.
    Номер строки -- индекс массива $a$.
    Номер столбца -- индекс элемента в определённом массиве из $a$.
    Одна и та же структура матрицы будет использована для хранения
    элементов любых типов.
    
    Вектор представляет собой структуру $Vector$ из  3 вещественных чисел --
    координат математического вектора в пространстве.
    Точка в пространстве, $Point$, также представляется своим радиус-вектором.
    Таким образом, с точкой можно будет работать так же, как и с вектором.
    Таким образом, описание полей и вектора, и точки выглядит так:
    \begin{itemize}
        \item вещественное число $x$;
        \item вещественное число $y$;
        \item вещественное число $z$.
    \end{itemize}
    
    Цвет состоит 3 вещественный числа $r, g, b$, обозначающих интенсивность
    красного, синего и зелёного цветов соответственно.
    Эти числа могут обозначать не только цвет,
    но и интенсивность составляющих света.
    Поэтому на значения полей не накладывается ограничений.
    Поля данной структуры $Color$:
    \begin{itemize}
        \item вещественное число $r$ -- количество красного;
        \item вещественное число $g$ -- количество зелёного;
        \item вещественное число $b$ -- количество синего.
    \end{itemize}
    
    Буфер источника является матрицей, содержащей элементы типа
    столкновение луча с объектом, <<Collision>>.
    Каждый элемент содержит:
    \begin{itemize}
        \item $color$ -- цвет пикселя;
        \item вещественное число $depth$ -- глубина точки;
        \item вектор нормали $n$ к точке сцены;
        \item цвет $light$ -- свет, дошедший до точки сцены;
        \item $x, y$ -- координаты $x, y$ точки сцены;
        \item $properties$ -- свойства объекта в данной точке;
        \item $delta$ -- точность сравнения глубин
        при определении байта затенения точки.
    \end{itemize}
    
    Структура изображения, $Image$, содержит матрицу, элементы
    которой являются объектами типа цвет.
    Можно хранить в изображении его размеры.
    Полный набор полей будет зависеть от реализации данной
    структуры в используемой библиотеке какого-либо языка.
    Но достаточно только матрицы.
    
    Камера, $Observer$, представляет структуру с параметрами наблюдателя сцены.
    Список полей структуры $Observer$:
    \begin{itemize}
        \item $x, y, z$ -- координатные вектора;
        \item $location$ -- точка положения камеры.
    \end{itemize}
    
    Источник света, $LightSource$, также, как и камера,
    имеет свою систему координат.
    Она нужна для определения ориентации буфера источника в пространстве
    при использовании алгоритма отражающих карт теней.
    Поля структуры источника:
    \begin{itemize}
        \item $x, y, z$ -- координатные вектора;
        \item $location$ -- точка положения источника;
        \item $color$ -- цвет, хранящий силу по каждой составляющей.
    \end{itemize}


    В программе использована полигональная модель.
    Каждый объект представляется набором треугольных граней.
    Список полей структуры треугольной грани $Triangle$:
    \begin{itemize}
        \item беззнаковое целое число $a$ --
        индекс точки, обозначающей первую вершину;
        \item беззнаковое целое число $b$ --
        индекс точки, обозначающей вторую вершину;
        \item беззнаковое целое число $c$ --
        индекс точки, обозначающей третью вершину;
        \item вектор нормали $n$ к данной грани;
        \item $equation$ -- уравнения плоскости, проходящей через данную грань.
    \end{itemize}
    
    При трассировке лучей вокруг объектов описываются сферы.
    Поэтому требуется отдельная структура $Sphere$ со следующими полями:
    \begin{itemize}
        \item $center$ -- точка, обозначающая центр;
        \item $radius$ -- радиус сферы.
    \end{itemize}
    
    Список полей структуры Объекта сцены -- $Object$:
    \begin{itemize}
        \item $points$ -- массив точек;
        \item $triangles$ -- массив граней;
        \item $properties$ -- свойства объекта.
    \end{itemize}
    
    Структура Сцены, $Stage$, представляет собой структуру для
    хранения всех данных, необходимых и достаточных для создания изображения.
    \begin{itemize}
        \item $sources$ -- массив источников света;
        \item $objects$ -- массив объектов;
        \item $camera$ -- камера.
    \end{itemize}
    
    Поля структуры данных, содержащей свойства объекта (<<ObjectProperties>>):
    \begin{itemize}
        \item $color$ -- цвет;
        \item $transmission$ -- коэффициент пропускания;
        \item $reflection$ -- коэффициент отражения;
        \item $optDensity$ -- оптическая плотность.
    \end{itemize}
    
        Поля структуры данных Трассировщика -- <<RayTracer>>:
     \begin{itemize}
        \item $data$ -- матрица лучей;
        \item $spheres$ -- сферы, описанные вокруг объектов;
        \item $transformers$ -- структуры для перехода к координатам источников;
        \item $sourceMaps$ -- буферы источников;
        \item $objects$ -- объекты;
        \item $sources$ -- источники света;
        \item $imgWidth, imgHeight$ -- размеры результирующего изображения;
        \item $method$ -- номер метода генерации теней;
        \item $cameraLocation$ -- положение камеры.
     \end{itemize}
    
    Луч представляет собой массив сегментов луча <<BeamSegment>>.
    Поля структуры данных <<BeamSegment>>:
    \begin{itemize}
        \item $direction$ -- направление луча;
         \item $origin$ -- исток луча;
        \item $color$ -- интенсивность каждой составляющей цвета;
        \item $collisionCount$ -- количество столкновений с объектами.
    \end{itemize}
    
    В алгоритме затенения с помощью карт теней производятся переходы
    к системам координат источников.
    Введена структура <<Transformer>> со следующими полями:
    \begin{itemize}
        \item $matrix$ -- матрица перехода к другой системе координат;
        \item $bias$ -- вектор, соединяющий начала новой и старой систем координат.
    \end{itemize}

    Структура $PlaneEquation$ представляет собой уравнения плоскости и содержит
    4 вещественных коэффициента $a, b, c, d$
    канонического уравнения данной плоскости.

    Введена структура $Line2d$, хранящая данные о прямой на плоскости.
    Она нужна в реализации алгоритма с буфером глубины.
    Поля структуры $Line2d$:
    \begin{itemize}
        \item $k, b$ -- вещественные коэфициенты уравнения прямой $y = kx + b$.
        \item $vartical$ -- информация о том, вертикальна ли прямая;
        \item $location$ -- координата $x$ положения прямой,
        если она вертикальна, то есть $k = \inf$
    \end{itemize}
}