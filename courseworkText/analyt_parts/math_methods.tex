\section{Преобразование объектов в трёхмерном пространстве} {
    Существует три вида преобразований объекта:
    поворот относительно одной из осей,
    перемещение и масштабирование.
    Для выполнения преобразования объекта преобразовывается
    каждая его вершина.
    Этого достаточно, так как треугольные грани хранятся
    как наборы трёх индексов в общем массиве вершин объекта.
    Рассмотрим преобразование точки $A = (x, y, z)$.
    Сначала формируется вектор $a = (x, y, z, 1)$.
    Затем он умножается на матрицу преобразования и получается новая точка.
    Еслил нужно применить к точке $A$ несколько последовательных преобразований,
    то нужно умножить матрицы этих преобразований в соответствующем порядке.
    После этого вектор $a$ умножается на полученную матрицу.
    Любые преобразования можно представить в виде комбинации поворота относительно оси,
    перемещения и масштабирования.
    Поэтому достаточно знать только матрицы, соответствующие этим
    базовым преобразованиям [5].
}

\subsection{Задачи трассировки лучей} {
    Рассмотрев алгоритм трассировки лучей, можно понять, что её реализация
    требует решения некоторых математических задач.
    Цель этого раздела -- предоставление методов для их решения.
    Также эти методы можно будет использовать
    в любом месте программы по необходимости.
    Далее представлены подзадачи трассировки и их решения, основанные на
    аналитической геометрии.
    
   \subsubsection{Составление уравнения плоскости} {
        Для нахождения точки пересечения луча с гранью объекта необходимо
        реализовать составление уравнения плоскости, содержащей грань, по 3 точкам.
        Плоскость $\alpha$ задаётся каноническим уравнением
        $Ax + By + Cz + D = 0$.
        Причём в этом уравнении коэффициенты $A, B, C$ -- координаты вектора нормали
        $\vec{n}$ к плоскости $\alpha$.
        Тогда задача сводится к нахождению $\vec{n}$.
        Известны три точки плоскости -- три вершины треугольника.
        Значит можно построить вектора $\vec{a}, \vec{b}$ на любых
        сторонах этого треугольника и $\vec{n} = \vec{a} \times \vec{b}$.
        После нахождения $A, B, C$ можно определить коэффициент $D$: \\
        $D = -Ax - By - Cz$, где $(x, y, z)$ -- любая вершина треугольника. \\
        Таким образом, вычисляются все коэффициенты канонического уравнения
        $\alpha$.
    }
    \subsubsection{Поиcк пересечения прямой с треугольной гранью} {
        Отслеживание хода луча содержит предполагает поиск
        его пересечение с какой-либо гранью.
        Сначала необходимо найти точку пересечения прямой и плоскости,
        в которой лежит грань, а затем определить,
        принадлежит ли данная точка самой грани.
        Прямая $a$ задана параметрически вектором $\vec{v}$ и точкой $O$.
        Вершины грани $P_1, P_2, P_3$ лежат в плоскости $\phi$.
        Рассмотрим способ нахождения пересечения $a$ с плоскостью $\phi$.
        Пусть $\phi$ задаётся уравнением $Ax + By + Cz = 0$,
        $\vec{n}$ -- единичный вектор нормали к данной плоскости,
        точка $P \in \phi$, а $\vec{n}$ -- вектор нормали к $\phi$.
        Тогда $\psi = |(\vec{OP}; \vec{n})|$ -- расстояние от $O$ до $\phi$.
        Если $(\vec{a}; \vec{n}) = 0$, то $\vec{a} \perp \vec{n} \implies$
        прямая $a$ не пересекает $\phi$, а значит, и саму грань.
        Поэтому перед началом вычислений необходимо выполнить данную проверку.
        Расстояние $\psi$ можно представить как длину проекции
        $\vec{OP}$ на $\vec{n}$.
        Пусть $O' = a \cap \phi$.
        Отношение модулей проекций векторов на одну прямую равно
        отношению длин этих векторов.
        Отсюда $\frac{|OO'|}{|\vec{v}|}=
        \frac{|(\vec{OO'};\vec{n})|}{|(\vec{v};\vec{n})|} = k$.
        Но $P$ $|(\vec{OP}; \vec{n})| = \psi = |(\vec{OO'}; \vec{n})|$.
        Тогда $k = \frac{|(\vec{v}, \vec{n})|}{\psi}$.
        Чтобы найти точку $O'$, нужно передвинуть точку $O$ на вектор $\vec{v} \cdot k$.
        Это следует из того, что $k = \frac{|OO'|}{|\vec{v}|}$.
        После того, как точка $O'$ найдена, выполняется проверка
        её принадлежности грани $P_1P_2P_3$.
        Таким образом решается поставленная задача.

        Предполагается, что вектор $\vec{n}$ и уравнения плоскостей известны заранее.
        Это уменьшит трудоёмкость вычислений.
        В итоге для нахождения пересечения параметрически заданной прямой и
        грани программе нужно будет выполнить следующие действия:
        \begin{itemize}
            \item найти скалярное произведение $\alpha$ векторов $\vec{v}$ и $\vec{n}$;
            \item если значение $|\alpha|$ близко к 0, то завершить алгоритм, так как пересечение отсутствует.
            \item выбрать на плоскости точку $P$ -- можно взять вершину грани;
            \item найти скалярное произведение $\psi$ векторов $\vec{a}$ и $\vec{n}$;
            \item вычислить значение $k = \frac{\alpha}{\psi}$;
            \item передвинуть точку $O$ вдоль вектора $k \cdot \vec{v}$ и получить $O'$;
            \item проверить принадлежность $O'$ грани $P_1P_2P_3$.
        \end{itemize}
    }
    \subsection{Определение принадлежности точки грани} {
        Описанный ранее метод содержит подзадачу определения
        принадлежности точки грани объекта.
        Далее описан метод её решения.
        Пусть площадь треугольника $t$ обозначается как $S(t)$
        Чтобы понять, принадлежит ли точка $P$ треугольнику $ABC$,
        нужно проверить равенство $S(ABC) = S(PAB) + S(PBC) + S(PAC)$.
        Если оно выполнено, то точка лежит внутри треугольника.
        Функция площади $S(t)$ может быть вычислена по трём сторонам
        треугольника по формуле Герона \ref{f:geron}, где $a, b, c$ -- стороны треугольника, а $p$ -- полупериметр.
        \begin{equation}
            \label{f:geron}
            S = \sqrt{p(p - a)(p - b)(p - c)}
        \end{equation}
    }
}

\section{Переход к другой системе координат} {
    Задача затенения с помощью отражательных карт теней содержит подзадачу
    перехода к другой системе координат [9].
    Рассматривается метод её решения, основанный на аналитической геометрии.
    Пусть задан базис $B = (\vec{x}, \vec{y}, \vec{z})$
    с началом координат в точке $O$,
    а вектора $\vec{i} = (1, 0, 0)^T, \vec{j} = (0, 1, 0)^T, \vec{k} = (0, 0, 1)^T$
    представляются как \ref{f:i}--\ref{f:k}.
    \begin{equation}
        \label{f:i}
        i = x_i \cdot \vec{x} + y_i \cdot \vec{y} + z_i \cdot \vec{z}
    \end{equation}
     \begin{equation}
        \label{f:j}
        j = x_j \cdot \vec{x} + y_j \cdot \vec{y} + z_j \cdot \vec{z}
    \end{equation}
     \begin{equation}
        \label{f:k}
        k = x_k \cdot \vec{x} + y_k \cdot \vec{y} + z_k \cdot \vec{z}
    \end{equation}
    Тогда матрица перехода $T$ к базису $B$ представляется в виде \ref{f:tr_mat}. \\
    \begin{equation}
        T = \begin{bmatrix}
            \label{f:tr_mat}
            x_i & x_j & x_k \\
            y_i & y_j & y_k \\
            z_i & z_j & z_k \\
        \end{bmatrix}.
    \end{equation}
    
    Координаты вектора $\vec{a}$ относительно базиса $B$ можно выразить по формуле \ref{f:vec_trans}.
    \begin{equation}
        \label{f:vec_trans}
        \vec{a'} = T \times \vec{a}
    \end{equation}
    Радиус-вектор точки $P$ относительно базиса $B$
    рассчитывается по формуле \ref{f:point_trans}.
    \begin{equation}
        \label{f:point_trans}
        \vec{P'} = -\vec{O} + (T \times \vec{P})
    \end{equation}
    Имея способ преобразования координат точки и вектора к другой системе координат,
    можно преобразовать к ней любой объект.
    Для этого достаточно выполнить преобразование всех вершин и
    векторов внешней нормали к граням.
}