\section{Преобразование объектов в трёхмерном пространстве} {
    Существует три вида преобразований объекта: поворот относительно одной из осей,
    перемещение и масштабирование.
    Для выполнения преобразования объекта преобразовывается каждая его вершина.
    Этого достаточно, так как треугольные грани хранятся как наборы трёх индексов
    в общем наборе вершин объекта.
    Рассмотрим преобразование точки $A = (x, y, z)$.
    Сначала формируется вектор $a = (x, y, z, 1)$.
    Затем он умножается на матрицу преобразования и получается новая точка.
    Еслил нужно применить к точке $A$ несколько последовательных преобразований,
    то нужно умножить матрицы этих преобразований в соответствующем порядке.
    После этого вектор $a$ умножается на полученную матрицу.
    Любые преобразования можно представить в виде комбинации попорота относительно оси,
    перемещения и масштабирования.
    Поэтому достаточно знать только матрицы, соответствующие этим
    базовым преобразованиям.
    Рассмотрим эти матрицы. \\
    \vspace{5mm}
    Поворот на угол $\phi$ относительно осей $x, y, z$ соответственно: \\
    \vspace{2mm}
    $
    \begin{bmatrix}
        1 & 0 & 0 & 0 \\
        0 & \cos\phi & -\sin\phi & 0 \\
        0 & \sin\phi & \cos\phi & 0 \\
        0 & 0 & 0 & 1 \\
    \end{bmatrix}
    $
    \hspace{5mm}
    $
    \begin{bmatrix}
        \cos\phi & 0 & \sin\phi & 0 \\
        0 & 1 & 0 & 0 \\
        -\sin\phi & 0 & \cos\phi & 0 \\
        0 & 0 & 0 & 1 \\
    \end{bmatrix}
    $
    \hspace{5mm}
    $
    \begin{bmatrix}
        \cos\phi & -\sin\phi & 0 & 0 \\
        \sin\phi & \cos\phi & 0 & 0 \\
        0 & 0 & 1 & 0 \\
        0 & 0 & 0  &1 \\
    \end{bmatrix}
    $
    \\
    \vspace{5mm}
    Перемещение: \\
    \vspace{2mm}
    $
    \begin{bmatrix}
        1 & 0 & 0 & t_x \\
        0 & 1 & 0 & t_y \\
        0 & 0 & 1 & t_z \\
        0 & 0 & 0 & 1 \\
    \end{bmatrix}
    $, где $\vec{t} = (t_x, t_y, t_z)$ - вектор перемещения. \\
    \vspace{5mm}
    
    Масштабирование: \\
    \vspace{2mm}
    $
    \begin{bmatrix}
        k_x & 0 & 0 & 0 \\
        0 & k_y & 0 & 0 \\
        0 & 0 & k_z & 0 \\
        0 & 0 & 0 & 1 \\
    \end{bmatrix}
    $, где $k_x, k_y, k_z$ - коэфициенты масштабирования. \\
    \vspace{5mm}
}

\subsection{Задачи трассировки лучей} {
   \subsubsection{Составление уравнения плоскости} {
        Плоскость $\alpha$ задаётся каноническим уравнением
        $Ax + By + Cz + D = 0$.
        Причём в этом уравнении коэфициенты $A, B, C$ - координаты вектора нормали
        к $\alpha$, то есть $\vec{n} = {A, B, C} \perp \alpha$.
        Тогда можно найти координаты вектора $\vec{n}$ и таким образом
        получить три коэфициента уравнения.
        Известны три точки плоскости -- три вершины треугольника.
        Тогда можно построить вектора $\vec{a}, \vec{b}$ на любых
        сторонах этого треугольника.
        Тогда $\vec{n} = \vec{a} \times \vec{b}$.
        Векторное произведение $\vec{a} \times \vec{b}$ можно
        определить с помощью нахождения определителя матрицы:
        $
        \begin{bmatrix}
            \vec{i} & \vec{j} & \vec{k} \\
            a_x & a_y & a_z \\
            b_x & b_y & b_z \\
        \end{bmatrix}
        $
        После нахождения $A, B, C$ можно определить коэфициент $D$: \\
        $D = -Ax - By - Cz$, где $(x, y, z)$ - любая вершина треугольника. \\
        Таким образом, вычисляются все коэфициенты уравнения.
    }
    \subsubsection{Поиcк пересечения прямой с треугольной гранью} {
        Сначала необходимо найти точку пересечения прямой и плоскости,
        в которой лежит грань, а затем определить,
        принадлежит ли данная точка самой грани.
        Прямая $a$ задана параметрически вектором $\vec{v}$ и точкой $O$.
        Вершины грани $P_1, P_2, P_3$ лежат в плоскости $\phi$.
        Рассмотрим способ нахождения пересечения $a$ с плоскостью $\phi$.
        Пусть $\phi$ задаётся уравнением $Ax + By + Cz = 0$,
        $\vec{n}$ -- единичный вектор нормали к данной плоскости,
        точка $P \in \phi$, а $\vec{n}$ -- вектор нормали к $\phi$.
        Тогда $\psi = |(\vec{OP}; \vec{n})|$ -- расстояние от $O$ до $\phi$.
        Если $(\vec{n}; \vec{n})$, то $\vec{n} \perp \vec{n} \implies$
        прямая $a$ не пересекает $\phi$, а значит, и саму грань.
        Поэтому перед началом вычислений необходимо выполнить соответствующую проверку.
        Расстояние $\psi$ можно представить как длину проекции
        $\vec{OP}$ на $\vec{n}$.
        Пусть $O' = a \cap \phi$.
        Отношение модулей проекций векторов на одну прямую равно
        отношению длин этих векторов.
        Отсюда $\frac{|OO'|}{|\vec{v}|}=
        \frac{|(\vec{OO'};\vec{n})|}{|(\vec{v};\vec{n})|} = k$.
        Но $P$ $|(\vec{OP}; \vec{n})| = \psi = |(\vec{OO'}; \vec{n})|$.
        Тогда $k = \frac{|(\vec{v}, \vec{n})|}{\psi}$.
        Чтобы найти точку $O'$, нужно передвинуть точку $O$ на вектор $\vec{v} \cdot k$.
        Это следует из того, что $k = \frac{|OO'|}{|\vec{v}|}$.
        После того, как точка $O'$ найдена, выполняется проверка
        её принадлежности грани $P_1P_2P_3$.
        Таким образом решается поставленная задача.
        \hspace{1.25cm}
        Предполагается, что вектор $\vec{n}$ и уравнения плоскостей известны заранее.
        Это уменьшит трудоёмкость вычислений.
        В итоге для нажождения пересечения параметрически заданной прямой и
        грани программе нужно будет выполнить седующие действия:
        \begin{enumerate}
            \item найти скалярное произведение $\alpha$ векторов $\vec{v}$ и $\vec{n}$;
            \item Если значение $|\alpha|$ близко к 0, то завершить алгоритм, так как пересечение отсутствует.
            \item выбрать на плоскости точку $P$ -- можно взять вершину грани;
            \item найти скалярное произведение $\psi$ векторов $\vec{a}$ и $\vec{n}$;
            \item вычислить значение $k = \frac{\alpha}{\psi}$;
            \item передвинуть точку $O$ вдоль вектора $k \cdot \vec{v}$ и получить $O'$;
            \item проверить принадлежность $O'$ грани $P_1P_2P_3$.
        \end{enumerate}
    }
    \subsection{Определение принадлежности точки треугольнику} {
        \hspace{1.25cm} Площадь треугольника $t$ пусть обозначается как $S(t)$
        Чтобы понять, принадлежит ли точка $P$ треугольгнику $ABC$
        нужно проверить равенство $S(ABC) = S(PAB) + S(PBC) + S(PAC)$.
        Если оно выполнено, то точка лежит внутри треугольника.
        Функция площади $S(t)$ может быть вычислена по трём сторонам
        треугольника по формуле Герона:
        $S = \sqrt{p(p - a)(p - b)(p - c)}$, \\
        где $a, b, c$ - стороны треугольника, а $p = \frac{a + b + c}{2}.$
    }
}

\section{Переход к другой системе координат} {
    Пусть задан базис $B = (\vec{x}, \vec{y}, \vec{z})$,
    а вектора $\vec{i} = (1, 0, 0)^T, \vec{j} = (0, 1, 0)^T, \vec{k} = (0, 0, 1)^T$
    представляются следующим образом:
    \begin{enumerate}
        \item $i = x_i \cdot \vec{x} + y_i \cdot \vec{y} + z_i \cdot \vec{z})$
        \item $j = x_j \cdot \vec{x} + y_j \cdot \vec{y} + z_j \cdot \vec{z})$
        \item $k = x_k \cdot \vec{x} + y_k \cdot \vec{y} + z_k \cdot \vec{z})$
    \end{enumerate}
    Тогда матрица перехода к базису $B$ равна
    $T =
    \begin{bmatrix}
        x_i & x_j & x_k \\
        y_i & y_j & y_k \\
        z_i & z_j & z_k \\
    \end{bmatrix}
    $
    Координаты вектора $\vec{a} = (x_a, y_a, z_a)^T$ относительно базиса $B$ можно выразить как $\vec{a_B} = T \times \vec{a}$.
    Далее рассмотрен способ нахождения координат точки $P(x, y, z)$ относительно
    системы координат с осями $\vec{i}, \vec{j}, \vec{k}$ и центром в точке
    $O(x_o, y_o, z_o)$:
    \begin{enumerate}
        \item $\vec{P'} = T \times \vec{P}$;
        \item $\vec{O}' = \vec{-O} + \vec{P'}$ -- радиус-вектор результирующей
        точки $O'(x_o', y_o', z_o')$.
    \end{enumerate}
    Имея способ преобразования координат точки и вектора к другой системе координат,
    можно преобразовать к ней любой объект.
}