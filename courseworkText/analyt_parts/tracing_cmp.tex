\section{Трассировка лучей} {
    Идея основана на том, что в реальном мире луч света, испускаемый источником,
    сначала отражается от объекта, а затем попадает в камеру или глаз человека.
    Таким образом происходит визуализация точки объекта.
    Чтобы определить, какую точку увидит наблюдатель в конкретном направлении, то есть луч от какой точки придёт к нему, нужно проследить за ходом этого луча
    в обратном направлении до тех пор, пока не будет обнаружен объект,
    от которого мог отразиться луч.
    Считается, что лучей настолько много, что всегда найдётся тот,
    который отражается в нужном направлении.
    Далее рассмотрены и сравнены самые популярные методы трассировки лучей~[10].
    Все они основаны на моделировании движения точки вдоль вектора $\vec{v}$ до
    столкновения с объектом сцены.
    Для сравнения использованы следующие критерии:
    \begin{itemize}
        \item отсутствие перебора всех полигонов сцены для каждого луча -- К1;
        \item гарантия учёта всех объектов -- К2.
    \end{itemize}

    \subsubsection{С помощью сфер} {
        Определяется расстояние $r$ от данной точки $p$ до ближайшего объекта~[10].
        Тогда в пределах сферы c центром в $p$ и радиусом $r$ нет никаких
        препятствий для распространения луча.
        Значит можно сделать шаг размером $r$ в направлении $\vec{v}$.
        Когда значение $r$ станет меньше определённого минимума, то можно считать,
        что обнаружено столкновение луча с объектом.
        Для оптимизации можно описать сферы вокруг объектов и, когда луч достигнет
        одной из сфер, искать пересечение с объектом внутри неё.
        В результате все полигоны сцены не перебираются и даже очень мелкие объекты
        учитываются.
    }

    \subsubsection{С помощью шагов} {
        Выполняется проход по лучу с фиксированным шагом $s$,
        на каждом из которых
        проверяется принадлежность точки какому-либо объекту.
        Если точка оказалась внутри объекта, то граница этого объекта лежит на
        отрезке между предыдущим и текущим положениями точки~[10].
        Тогда, если $s \geq s_{min}$, точка возвращается на шаг назад и $s$
        уменьшается в 2 раза, иначе фиксируется столкновение луча с объектом.
        Данный метод трассировки лучей имеет недостаток:
        шаг $s$ может оказаться таким, что точка
        пройдёт мимо маленького объекта и он не будет отображён~[10].
        Следовательно, критерий К2 не выполнен.
        Чтобы зафиксировать принадлежность точки объекту, нужно перебирать
        полигоны сцены на каждом шаге.
        Поэтому критерий K1 тоже не удовлетворён.
    }
    
    \subsubsection{С помощью перебора полигонов} {
        Луч пересекается с каждым полигоном сцены, после чего определяется ближайшее
        из найденных пересечений к началу данного луча~[10].
        Чтобы получить возможность избежать перебора полигонов сцены, можно содержать
        список активных граней.
        Грань активна на данном шаге, если она участвует в переборе.
        Чтобы определить, какая грань активна, нужно описать прямоугольник вокруг
        её проекции и ответить на вопрос, попадает ли луч на неё.
        Но данный метод хорошо работает, когда луч перпендикулярен
        картинной плоскости.
        А в данной работе следует учесть отражения лучей.
        Поэтому перебор полигонов сцены будет присутствовать.
        К достоинствам алгоритма можно отнести независимость трудоёмкости от расстояния между телами и гарантию учёта даже очень маленьких объектов.
    }

    В таблице 4 представлен краткий вывод по соответствию критериям.
    
    \begin{center}
        \begin{tabular} { |c|c|c| }
            \hline
            \hspace{0pt} & \multicolumn{2}{|c|}{Критерий} \\
            \hline
            Алгоритм & К1 & К2 \\
            \hline
            С помощью сфер & + & + \\
            \hline
            С помощью шагов & + & - \\
            \hline
            С помощью перебора полигонов & - & + \\
            \hline
        \end{tabular}
        \\
    \vspace{2mm}
    \small { Таблица 4 -- сравнение алгоритмов трассировки лучей }
    \end{center}

    \section*{Выводы} {
        Из описанных способов наиболее удовлетворяет критериям только
        один, который основан на нахождении расстояния до ближайших сфер
        и последующем поиске пересечения с объектом.
        Его и следует использовать в программе.
    }
}