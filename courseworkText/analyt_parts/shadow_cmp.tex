\section{Выбор алгоритма отрисовки теней} {
    Далее рассмотрены и сравнены основные алгоритмы отрисовки теней [9].
    Сравнение произведено по сдедующим критериям:
    \begin{itemize}
        \item возможность учёта уменьшения интенсивности света с ростом
        расстояния до источника -- К1;
        \item возможность учёта диффузное отражение света
        от поверхности других объектов -- К2;
        \item возможность учёта уменьшение интенсивности тени из-за
        прохождения света через объекты -- K3.
    \end{itemize}
    
    \subsection{Распространение по объёму} {
        Данный алгоритм основан на моделировании прохождения световой волны по сцене.
        Пространство делится на маленькие кубы -- ячейки [9].
        Свет распространяется из ячейки с источником до всех остальных.
        На каждом шаге интенсивность света данной ячейки распределяется между
        соседними и, если она не содержит объектов, удаляется из памяти.
        Яркость кусочка тела, находящегося в ячейке, будет равна её яркости.
        Прохождение света через объект также можно учесть, если при столкновении волны света с объектом не останавливать её, а ркгелировать яркость.
    }
    \subsection{Список приоритетов (алгоритм художника)} {
        В данном методе производится сортировка граней объектов
        как в алгоритме художника, после чего они сортируются [9].
        При отрисовке каждой грани сначала находится и отрисовывается
        многоугольник, обозначающий её тень, отброшенную на плоскость
        проецирования сцены, а затем многоугольник самой грани.
        Под отрисовкой тени подразумевается уменьшение яркости изображённых пикселей.
        В результате получается изображение сцены со всеми тенями, а невидимые тени,
        также как и невидимые грани, покрываются другими элементами.
        Также возможно учесть уменьшение интенсивности света с ростом
        расстояния до точки с помощью построения плоскостей через грани.
        Недостатки относительно выделенных критериев заключаются в отсутствии учёта
        диффузного отражения света и прохождения света через объекты.
    }
    \subsection{Карта теней} {
        В данном алгоритме строится 2 буфера глубины:
        $buf_1$ относительно камеры и $buf_2$ относительно источника [9].
        Пусть точка $P$ соответствует элементам $a, b$
        буферов $buf_1, buf_2$ соответственно.
        Находится глубина $d$ точки $P$ относительно источника.
        $P$ находится в тени, если $b < d$.
        Иначе можно вычислить её яркость исходя из значения $b$.
        В результате можно определить только яркость изображаемой точки, полученную от источника и составить карту теней -- матрицу размерности изображения,
        содержащую информацию о яркости соответствующего пикселя.
        В результате удовлетворён только первый критерий.
    }
    \subsection{Отражательная карта теней} {
        Является модификацией алгоритма с картой теней [2].
        Считается, что точки сцены способны отражать свет от источника, обеспечивая
        дополнительную яркость других точек.
        Сначала определяется освещена ли точка источником напрямую.
        Затем выбираются точки, вносящие вклад в освещение данной и вычисляется
        дополнительная яркость от диффузного отражения.
        Это достигается засчёт хранения дополнительной информации в буфере глубины.
        Таким образом, можно учесть диффузное отражение.
        Также разработано улучшение данного алгоритма.
        В элементах буфера глубины хранится целый массив структур, каждая из которых
        описывает соответствующую точку сцены.
        Эти структуры отсортированы по глубине точек и для каждой расчитана яркость.
        В результате появляется возможность анализировать яркость любой точки сцены
        с учётом прохождения лучей через объекты.
        Единственное упущение связано с отсутствием рассмотрения преломления
        луча при прохождении через объект.
        Но данная особенность не существенна.
    }

    \subsection {Трассировка лучей} {
        Для определения яркости точки объекта есть метод, основанный на
        трассировке [10].
        Его идея заключается в отслеживании хода луча от точки объекта до
        источника.
        Если этот луч не встретил препятствий, то данная точка освещена
        источником напрямую, а если столкнулся с другим полигоном, то итоговая
        яркость точки умножается на коэффициент пропускания.
        Таким образом, возможность учесть уменьшение интенсивности света с
        ростом расстояния до источника и прохождение света через объекты.
    }

    \subsection {Модель Фонга} {
        Существует способ вычисления яркости любой точки сцены,
        полученной от источника света.
        Пусть:
        \begin{itemize}
            \item луч падает на точку объекта $p$;
            \item
                $\Phi$ -- интенсивность света от источника,
                дошедшая до данной точки;
            \item $\vec{v}$ -- единичный вектор направления к наблюдателю;
            \item $\vec{l}$ -- единичный вектор направления к источнику;
            \item $\vec{r}$ -- единичный вектор направления отражённого луча;
            \item $\vec{n}$ -- единичный вектор нормали к поверхности;
            \item $k_d$ -- коэффициент диффузного отражения;
            \item $k_s$ -- коэффициент зеркального отражения;
            \item $\alpha$ -- коэффициент блеска;
            \item $I_b = const$ -- фоновое освещение;
            \item $I_d$ -- диффузная яркость;
            \item $I_s$ -- зеркальная яркость;
            \item $I$ -- итоговая яркость.
        \end{itemize}
        Тогда выполнены уравнения \ref{f:I_d}--\ref{f:I} [6].
        \begin{equation}
            \label{f:I_d}
            I_d = k_d(\vec{l}, \vec{n}) \cdot \Phi
        \end{equation}
        \begin{equation}
            \label{f:I_s}
            I_s = k_s(\vec{r}, \vec{v})^{\alpha} \cdot \Phi
        \end{equation}
        \begin{equation}
            \label{f:I}
            I = I_b + I_d + I_s
        \end{equation}
        Значение $\Phi$ вычисляется путём трассировки луча
        от точки до источника света [6].
        Таким образом, возможно предусмотреть прохождение света через
        объекты и уменьшение яркости луча с расстоянием до источника,
        но нельзя учесть освещение от отражения света от других объектов.
        Каждая составляющей цвета вычисляется отдельно [6].
    }
    В таблице 3 представлен краткий вывод по соответствию критериям.
    \begin{center}
        \begin{tabular} { |c|c|c|c|c| }
            \hline
            \hspace{0pt} & \multicolumn{3}{|c|}{Критерий} \\
            \hline
            Алгоритм & К1 & К2 & К3 \\
            \hline
            Распространение по объёму & + & - & + \\
            \hline
            Со списком приоритетов & + & - & -  \\
            \hline
            Карта теней & + & - & - \\
            \hline
            Отражательная карта теней & + & + & + \\
            \hline
            Трассировка лучей & + & - & + \\
            \hline
            Модель Фонга & + & - & + \\
            \hline
        \end{tabular}
        \\
        \vspace{2mm}
        \small { Таблица 3 -- сравнение алгоритмов наложения теней }
    \end{center}
    
    \section*{Выводы} {
        Среди рассмотренных алгоритмов генерации теней, отражательная карта теней
        удовлетворяет наибольшему количеству критериев.
        Поэтому данный метод выбран для реализации в программе.
        Трассировка лучей и модель Фонга удовлетворяют 2 из 3 критериев, поэтому
        было решено реализовать в том числе их для сравнения результатов
        затенения.
        Распространение по объёму предполагает воксельное представление модели,
        так как яркость, достигшую каждой точки от источника тоже нужно хранить.
        Поэтому экспериментов с данным алгоритмом не будет.
    }
}