\section{Выбор алгоритма отрисовки теней} {
    Далее рассмотрены и сравнены самые популярные алгоритмы отрисовки теней.
    Сравнение произведено по сделующим критериям:
    \begin{enumerate}
        \item возможность учесть уменьшения интенсивности света с ростом расстояния
        до источника -- К1;
        \item возможность учесть диффузное отражение света
        от поверхности других объектов -- К2;
        \item возможность учесть уменьшение интенсивности тени из-за
        прохождения света через объекты -- K3.
    \end{enumerate}
    
    \subsection{Распространение по объёму} {
        \hspace{1.25cm}
        Данный алгоритм основан на моделировании прохождения световой волны по сцене.
        Пространство делится на маленькие кубы -- ячейки.
        Свет распространяется из ячейки с источником до всех остальных.
        На каждом шаге интенсивность света данной ячейки распределяется между
        соседними и, если она не содержит обектов, удаляется из памяти.
        Яркость кусочка тела, находящегося в ячейке, будет равна её яркости.
        Прохождение света через объект также можно учесть, если при столкновении волны света с объектом не останавливать её, а ркгелировать яркость.
    }
    \subsection{Список приоритетов (алгоритм художника)} {
        \hspace{1.25cm}
        В данном методе производится сортировка граней объектов
        как в алгоритме художника, после чего они сортируются.
        При отрисовке каждой грани сначала находится и отрисовывается
        многоугольник, обозначающий её тень, отброшенную на плоскость
        проецирования сцены, а затем многоугольник самой грани.
        Под отрисовкой тени подразумевается уменьшение яркости изображённых пикселей.
        В результате получается изображение сцены со всеми тенями, а невидимые тени,
        также как и невидимые грани, покрываются другими элементами.
        Также возможно учесть уменьшение интенсивности света с ростом
        расстояния до точки с помощью построения плоскостей через грани.
        Недостатки относительно выделенных критериев заключаются в отсутствии учёта
        диффузного отражения света и прохождения света через объекты.
    }
    \subsection{Карта теней} {
        \hspace{1.25cm}
        В данном алгоритме строится 2 буфера глубины:
        $buf_1$ относительно камеры и $buf_2$ относительно источника.
        Пусть точка $P$ соответствует элементам $a, b$
        буферов $buf_1, buf_2$ соответственно.
        Находится глубина $d$ точки $P$ относительно источника.
        $P$ находится в тени, если $b < d$.
        Иначе можно вычислить её яркость исходя из значения $b$.
        В результате можно определить только яркость изображаемой точки, полученную от источника и составить карту теней -- матрицу размерности изображения,
        содержащую информацию о яркости соответствующего пикселя.
        В результате удовлетворён только первый критерий.
    }
    \subsection{Отражательная карта теней} {
        \hspace{1.25cm}
        Является модификацией алгоритма с картой теней.
        Считается, что точки сцены способны отражать свет от источника, обеспечивая
        дополнительную яркость других точек.
        Сначала определяется освещена ли точка источником напрямую.
        Затем выбираются точки, вносящие вклад в освещение данной и вычисляется
        дополнительная яркость от диффузного отражения.
        Это достигается засчёт хранения дополнительной информации в буфере глубины.
        Таким образом, можно учесть диффузное отражение.
        Также разработано улучшение данного алгоритма.
        В элементах буфера глубины хранится целый массив структур, каждая из которых
        описывает соответствующую точку сцены.
        Эти структуры отсортированы по глубине точек и для каждой расчитана яркость.
        В результате появляется возможность анализировать яркость любой точки сцены
        с учётом прохождения лучей через объекты.
        Единственное упущение связано с отсутствием рассмотрения преломления
        луча при прохождении через объект.
        Но данная особенность не существенна.
    }

    \section {Трассировка лучей} {
        Для определения яркости точки объекта есть метод, основанный на
        трассировке.
        Его идея заключается в отслеживании хода луча от точки объекта до
        источника.
        Если этот луч не встретил препятствий, то данная точка освещена
        источником напрямую, а если столкнулся с другим полигоном, то итоговая
        яркость точки умножается на коэфициент пропускания.
        Таким образом, возможность учесть уменбшение интенсивности света с
        ростом расстояния до источника и прохождение света через объекты.
    }
    \vspace{5mm}
    В таблице 2 представлен краткий вывод по соответствию критериям.
    \begin{center}
        \begin{tabular} { |c|c|c|c|c| }
            \hline
            \hspace{0pt} & \multicolumn{3}{|c|}{Критерий} \\
            \hline
            Алгоритм & К1 & К2 & К3 \\
            \hline
            Распространение по объёму & + & - & + \\
            \hline
            Со списком приоритетов & + & - & -  \\
            \hline
            Карта теней & + & - & - \\
            \hline
            Отражательная карта теней & + & + & + \\
            \hline
            Трассировка лучей & + & - & + \\
            \hline
        \end{tabular}
        \\
        \vspace{2mm}
        \small { Таблица 2 -- сравнение алгоритмов наложения теней }
    \end{center}
    
    \hspace{1.25cm}
    Среди рассмотренных алгоритмов генерации теней, отражательная карта теней
    удовлетворяет наибольшему количеству критериев.
    Поэтому данный метод выбран для реализации в программе.
}