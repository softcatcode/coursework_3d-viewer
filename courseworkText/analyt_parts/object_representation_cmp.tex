\section{Выбор способа представления объектов} {
    Далее рассмотрены и сравнены основные методы хранения объектов~[7].
    Для сравнения использованы следующие критерии:
    \begin{itemize}
        \item никакая информация о форме и свойствах объекта
        не дублируется -- К1;
        \item возможность представления невыпуклых объектов -- К2;
        \item возможность сведения задачи пересечения луча с объектом
        к пересечению прямой и плоскости -- К3.
    \end{itemize}
    \subsection{Конструкторская геометрия} {
        Данный способ представления основан на комбинировании примитивов.
        Сложные, в том числе невыпуклые объекты создаются с помощью операций пересечения, объединения и разности~[7].
        Таким образом, объект представляется в виде дерева,
        где листья -- выделенные элементарные объекты, а остальные узлы --
        теоретико-множественные операции.
        Такая модель не содержит дублирующейся информации, следовательно, критерий
        К1 также удовлетворён.
        Чтобы найти пересечение луча с объектом, нужно рассматривать его
        пересечения с элементарными объектами, а не с плоскими гранями,
        поэтому критерий К3 не удовлетворён.
    }
    \subsection{Воксельное представление} {
        Пространство делится на маленькие кубы -- воксели, для
        каждого из которых хранится информация о том, чем он заполнен~[7].
        Для экономии памяти все свойства объекта можно сохранить в отдельной
        структуре, а для вокселей хранить только указатель на такую структуру.
        Чем меньше воксели, тем точнее будет представлено тело.
        В результате можно представлять тела любой формы,
        но есть дублирующаяся информация.
    }
    \subsection{Аналитический метод} {
        Тело представляется как плоская фигура и уравнение её движения.
        С помощью параллельного переноса или вращения плоской фигуры
        можно задать сложное тело~[7].
        Таким образом, хранится фигура и функция от какого-либо параметра,
        результат которой -- координаты какой-либо точки этой фигуры.
        Особенность данного алгоритма в том, что для каждого тела нужно определять
        траекторию движения базовой фигуры.
        Это делает представление некоторых объектов очень сложным, хотя оно возможно.
    }
    \subsection{Плоские примитивы} {
        В данном способе хранится только оболочка объекта.
        Тело разбивается на простые фигуры~[7].
        Например, точки, рёбра и треугольники.
        Причём хранить рёбра при наличии граней не обязательно, так как
        каждое из них принадлежит какому-либо треугольнику.
        Если тело нельзя точно представить с помощью выделенных примитивов,
        то оно аппроксимируется~[7].
        В качестве плоской фигуры чаще выбирается треугольник, так как все его вершины всегда лежат в одной плоскости.
        Чтобы сэкономить память, хранится массив вершин объекта и массив треугольных граней, каждая из которых представляет собой 3 индекса в массиве вершин.
        В данном методе никакая информация не дублируется, выпуклые объекты могут
        быть представлены, и критерий К3 удовлетворён.
    }

    В таблице 1 представлен краткий вывод по соответствию критериям: \\
    \begin{center}
        \begin{tabular} { |c|c|c|c| }
            \hline
            \hspace{0pt} & \multicolumn{3}{|c|}{Критерий} \\
            \hline
            Алгоритм & К1 & К2 & К3 \\
            \hline
            Конструкторская геометрия & + & + & - \\
            \hline
            Воксельное представление & - & + & - \\
            \hline
            Аналитический метод & + & + & - \\
            \hline
            Плоские примитивы & + & + & + \\
            \hline
        \end{tabular} \\
        \vspace{2mm}
        \small { Таблица 1 -- Сравнение способов представления объектов. }
    \end{center}

    \section*{Выводы} {
        В результате выбран способ на основе плоских примитивов,
        в качестве которых взяты треугольники.
        Так как именно он удовлетворяет всем выделенным критериям.
    }
}