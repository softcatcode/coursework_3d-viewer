\section{Формализация технического задания}
В данной работе необходимо реализовать одновременную визуализацию тел.
Причём к программе предъявлен ряд требований:
\begin{itemize}
    \item тела должны задаваться описателями сетки;
    \item возможность отрисовки невыпуклых тел;
    \item
        коэффициенты диффузного и зеркального отражения,
        коэффициенты пропускания и преломления,
        а также оптическая плотность объекта и среды
        должны задаваться пользователем на этапе выполнения.
\end{itemize}
В соответствии с поставленной задачей, объект сцены представляет собой
структуру данных, состоящую из описателя сетки и свойств.
Под описателем сетки подразумевается структура данных, хранящая
информацию о форме поверхности объекта и его положении в пространстве.
Свойства объекта -- совокупность следующих значений:
цвет, коэффициенты зеркального и диффузного отражения, пропускания,
преломления и оптическую плотность материала.
Цвет представляется 3 целыми числами: количество красного, синего и зелёного,
а остальные параметры -- вещественными числами.
Из того, какие свойства тел требуется учесть, следует необходимость
задания источников света.
Их можно представить как совокупность направления и интенсивности освещения.
Для того чтобы изображение тел на освещённой сцене выглядело реалистично,
следует учесть наличие теней, так как неосвещённые участки сцены должны
выглядеть менее ярко.
Чтобы уменьшенить яркость пикселя,
достаточно уменьшить составляющие его цвета.
Для правильной отрисовки теней можно использовать один
из существующих алгоритмов затенения.
Таким образом, техническое задание формализовано.