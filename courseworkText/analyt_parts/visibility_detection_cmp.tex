\section{Выбор алгоритма удаления невидимых поверхностей} {
    Рассмотрим и сравним самые часто используемые алгоритмы удаления
    невидимых поверхностей.

    Далее описаны критерии сравнения:
    \begin{enumerate}
        \item возможность учесть прохождения света через некоторые объекты -- K1;
        \item возможность учесть преломления лучей -- K2;
        \item возможность учесть отражения лучей -- K3;
        \item отсутствие необходимости обрабатывать наблагоприятные
        ситуации отдельно -- K4;
        \item возможность обработки невыпуклых тел без их разбиения
        на выпуклые части -- K5.
    \end{enumerate}

    Теперь следует поверхностно рассмотреть сами алгоритмы с целью определить,
    каким критериям они удовлетворяют.
    В каждом случае предполагается, что объекты представлены в виде полигональной
    модели, так как этот метод выбран для реализации.

    \subsection{Буфер глубины} {
        В этом алгоритме предполагается, что вектор наблюдения сцены
        направлен по оси $z$.
        Под глубиной точки понимается её $z$ координата.
        Буфер глубины -- матрица размерности изображения, хранящая
        в своих элементах цвет и глубину изображаемой точки сцены.
        По номерам строки и столбца элемента буфера можно однозначно определить
        координаты $x, y$ точки, информацию о которой он хранит.
        Если элемент $(i, j)$ заполненного буфера содержит информацию
        о точке $(x_0, y_0, z_0)$, то эта точка имеет наименьшую глубину
        среди всех удовлетворяющих условию $x = x_0, y = y_0$.
        Таким образом, идея данного метода предполагает только прямолинейное
        распространение света до первого столкновения с поверхностью,
        но работает для объектов любой формы.
        При этом сцена никогда не требует предобработки.
    }

    \subsection{Алгоритм художника} {
        В данном алгоритме предполагается, что на сцене отсутствуют циклические
        наложения граней объектов.
        Если есть пересекающиеся грани, следует разбить их на непересекающиеся.
        Таким образом, алгоритм хдожника требует предобработки неблагоприятных
        случаев, но выпуклость объектов не важна,
        так как вся работа проводться над многоугольниками.
        Затем грани сортируются от самой дальней от наблюдателя до самой ближней.
        После этого их можно последовательно отрисовать.
        В результате получается готовое изображение сцены.
        Преломление, отражение и пропускание света не учитывается.
    }

    \subsection{Трассировка лучей} {
        Данный алгоритм основан на моделировании распространения луча в
        обратном направлении: от наблюдателя до объектов.
        Таким образом, имеется возможность менять направление этого луча
        после столкновения с поверхностью, а также разделять его на несколько других.
        Поэтому и преломление, и отражение, и пропускание света учтитывается.
        Также объекты могут быть невыпуклыми и сцена не требует предобработки.
    }
    
    В таблице 1 представлен краткий вывод по соответствию критериям.
    
    \begin{center}
        \begin{tabular} { |c|c|c|c|c|c| }
            \hline
            \hspace{0pt} & \multicolumn{5}{|c|}{Критерий} \\
            \hline
            Алгоритм & К1 & К2 & К3 & К4 & К5 \\
            \hline
            Буффер глубины & - & - & - & + & + \\
            \hline
            Алгоритм художника & - & - & - & - & +  \\
            \hline
            Трассировка лучей & + & + & + & + & + \\
            \hline
        \end{tabular}
        \\
        \vspace{2mm}
        \small { Таблица 1 -- сравнение алгоритмов удаления невидимых поверхностей }
    \end{center}

    В итоге определено, что трассировка лучей -- единственный из рассмотреных алгоритмов, который явялется достаточным для достижения цели работы.
    Так как именно он удовлетворяет критериям К1, К2, К3.
    Поэтому в программе следует использовать трассировку лучей.
}