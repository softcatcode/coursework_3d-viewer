\chapter*{Список использованных источников}
\addcontentsline{toc}{chapter}{Список использованных источников}

\hspace{-1.25cm}
1. Алгоритмы затенения: карты теней.
[Электронный ресурс] -- \\
{https://learnopengl.com/Advanced-Lighting/Shadows/Shadow-Mapping} \\
(дата обращения 17.07.2023) \\

\hspace{-1.25cm}
2. Алгоритмы затенения: отражательные карты теней (RSM).
[Электронный ресурс] -- \\
{https://ericpolman.com/2016/03/17/reflective-shadow-maps}, \\
{http://www.klayge.org/material/3\_12/GI/rsm.pdf} \\
(дата обращения 10.08.2023) \\

\hspace{-1.25cm}
3. Алгоритмы удаления невидимых поверхностей: конспект лекций.
[Электронный ресурс] -- \\
{http://www.graph.unn.ru/rus/materials/CG/CG13\_HSROptimization.pdf} \\
(дата обращения 21.07.2023) \\

\hspace{-1.25cm}
4. Алгоритмы генерации теней: статья о LightPropagation Volumes.
[Электронный ресурс] -- \\
{https://ericpolman.com/2016/06/28/light-propagation-volumes/} \\
(дата обращения 20.08.2023) \\

\hspace{-1.25cm}
5. Матрицы преобразований: конспект лекций по аналитической геометрии в
МГТУ им. Н. Э. Баумана каферы математического моделирования (ФН-12). \\

\hspace{-1.25cm}
6. Алгоритмы трассировки: статья о модели Фонга.
[Электронный ресурс] -- \\
{https://compgraphics.info/3D/lighting/phong\_reflection\_model.php} \\
(дата обращения 10.12.2023) \\

\hspace{-1.25cm}
7. Способы представления трёхмнрных объектов:
конспект лекций по компьютерной графике в МГТУ им. Н. Э. Баумана
каферы программной инфенерии (ИУ-7). \\

\hspace{-1.25cm}
8. Методы удаления невидимых поверхностей:
конспект лекций по компьютерной графике в МГТУ им. Н. Э. Баумана
каферы программной инфенерии (ИУ-7). \\

\hspace{-1.25cm}
9. Методы генерации теней:
конспект лекций по компьютерной графике в МГТУ им. Н. Э. Баумана
каферы программной инфенерии (ИУ-7). \\

\hspace{-1.25cm}
10. Методы трассировки лучей:
конспект лекций по компьютерной графике в МГТУ им. Н. Э. Баумана
каферы программной инфенерии (ИУ-7). \\