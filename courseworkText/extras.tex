\chapter{Приложение}

\section{Приложение A}

\lstinputlisting[
    label=lst:structures,
    caption=Объявление структур программы.
] {./code/structures.cpp}

\lstinputlisting[
    label=lst:buildShadowMaps,
    caption=Реализация построения буферов теней.
] {./code/shadowing/buildShadowMaps.cpp}

\lstinputlisting[
    label=lst:rsm,
    caption=
        Реализация вычисления яркости в алгоритме
        с отражательной картой теней.
] {./code/shadowing/rsmBrightness.cpp}
    
\lstinputlisting[
    label=lst:trace,
    caption=Реализация вычисления яркости с помощью трассировки.
] {./code/shadowing/traceBrightness.cpp}

\lstinputlisting[
    label=lst:fong,
    caption=Реализация вычисления яркости с помощью модели Фонга.
] {./code/shadowing/fongBrightness.cpp}

\lstinputlisting[
    label=lst:split,
    caption=Реализация разбиения луча на преломлённый и отражённый.
] {./code/tracing/split.cpp}

\lstinputlisting[
    label=lst:trace,
    caption=Реализация разбиения луча на преломлённый и отражённый.
] {./code/tracing/trace.cpp}

\lstinputlisting[
    label=lst:findNextCollision,
    caption=Реализация функции нахождения координат столкновения луча и объекта.
] {./code/tracing/findNextCollision.cpp}

\section {Приложение Б}
К работе прилагается презентация на 16 слайдов.