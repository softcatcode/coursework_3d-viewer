\chapter*{ЗАКЛЮЧЕНИЕ}
\addcontentsline{toc}{chapter}{ЗАКЛЮЧЕНИЕ}

В результате данной работы было выполнено следующее:
\begin{itemize}
    \item выбран алгоритм удаления невидимых поверхностей;
    \item выбраны 3 алгоритма генерации теней;
    \item выбран способ представления невыпуклых тел;
    \item составлена idef схема работы программы уровней 0 и 1;
    \item составлена схема алгоритма трассировки;
    \item
        написан код программы и получены изображения
        сцены с тремя способами затенения с помощью
        структурного программирования;
    \item было проанализировано различие результатов затенения в каждом случае;
    \item
        построены графики зависимости времени
        генерации изображения от его размеров для каждого
        реализованного алгоритма затенения, причём рассмотрены разные сцены;
    \item
        сделан вывод о том, что реализация алгоритма затенения
        с помощью карты теней самая быстрая,
        а с помощью модели Фонга -- самая медленная.
\end{itemize}
Таким образом, все поставленные задачи решены.
Следовательно, цель работы достигнута.